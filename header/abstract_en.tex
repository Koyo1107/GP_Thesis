\begin{center}
\textbf{\large Abstract of Bachelor's Thesis Academic Year 2020}

\vspace{6mm}

\textbf{\large Depth2Jam : Congestion Estimation System Using a Drive Recorder}

\end{center}

\vspace{10mm}


\begin{flushleft}
\textbf{Abstract}\\
\end{flushleft}
In Japan, traffic jam information is mainly obtained from sensors and monitoring cameras installed on roads, and sent to traffic information management centers such as VICS, which in turn send the information to car navigation systems.
However, traffic jam information can currently be obtained only on some roads, such as expressways and national ways, and cannot be collected on other roads.
In addition, the installation rate of drive recorders has been increasing as a result of recent improvements in performance and the surfacing of dangerous and incendiary driving.
However, there are few cases in which the videos from the installed drive recorders are used for viewing.
Since the drive recorder is a video device, it can collect various data by detecting objects and recognizing images.
The purpose of this research is to propose a new way to use the drive recorder and reducing the area where traffic jam information can be collected, which is limited.
As an approach to the purpose, we propose Depth2Jam, a system to estimate traffic jam from the images of drive recorders.
In this study, we implemented Depth2Jam and conducted an experimental evaluation of the accuracy of traffic jam estimation.
In addition, we show the improvement of the accuracy of Depth2Jam and its future prospects.

\begin{flushleft}
\textbf{Keywords}\\
\textbf{Drive Recorder,Traffic Jam, Deep Learning, Image Processing}
\end{flushleft}

\begin{flushright}
\textbf{Keio University Faculty of Environment and Infomation Studies.}\\
\textbf{Koyo Ri}\\
\end{flushright}
