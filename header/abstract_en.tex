\begin{center}
\textbf{\large Abstract of Bachelor's Thesis Academic Year 2020}

\vspace{6mm}

\textbf{\large Depth2Jam : Congestion Estimation System Using a Drive Recorder}

\end{center}

\vspace{10mm}


\begin{flushleft}
\textbf{Abstract}\\
\end{flushleft}
Congestion information in Japan is mainly calculated in real time from road sensors and surveillance camera images, and such information is used in car navigation systems. However, such systems are not installed on some roads and information is limited. This study aims to take advantage of the recent increase in the number of drive recorders and improvements in video technology to estimate traffic congestion on all roads by using monocular camera images obtained from drive recorders. The implementation uses monocular camera video from drive recorders, a distance estimation library and an object detection system to estimate whether or not a vehicle is congested based on the distance from the vehicle ahead.

\begin{flushleft}
\textbf{Keywords}\\
\textbf{}
\end{flushleft}

\begin{flushright}
\textbf{Keio University Faculty of Environment and Infomation Studies.}\\
\textbf{Koyo Ri}\\
\end{flushright}
