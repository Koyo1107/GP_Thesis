\begin{center}
\textbf{\Large 卒業論文要旨 2019年度(令和元年度)}

\vspace{6.18mm}

\textbf{\Large Depth2Jam:ドライブレコーダーを用いた渋滞推定システム}
\end{center}

\vspace{10mm}

\begin{flushleft}
\textbf{論文要旨}\\
\end{flushleft}
日本における渋滞情報は主に道路に設置されているセンサーや監視カメラの映像等からリアルタイムに算出されており,自動車のナビゲーションシステム等で使われている.
しかし現状渋滞情報を取得できるのは高速道路や国道といった一部の交通量が多い道路に限られている.
本研究では,近年のドライブレコーダー搭載数の増加や映像技術向上を生かし,ドライブレコーダーから得られる単眼カメラ映像から渋滞推定を行うことで,全ての道路での渋滞推定を可能にすることを目指す.
実装にはドライブレコーダーから得られる単眼カメラ映像を用いて,距離推定ライブラリ及び物体検出システムを利用し,先行車との距離をもとに渋滞しているか否かを推定する.
本研究が実現すると,現在のVICS等の渋滞,交通量を検知,測定するシステムでは賄えないような,細い道,住宅街の道,一方通行など,さまざまな道路で突発的に発生した渋滞情報を共有することができる.


\begin{flushleft}
\textbf{キーワード}\\
\textbf{ドライブレコーダー,渋滞,深層学習,画像処理}

\end{flushleft}

\begin{flushright}
\textbf{慶應義塾大学 環境情報学部}\\
\textbf{李 広耀}
\end{flushright}
\newpage

