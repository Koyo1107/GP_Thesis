\begin{center}
\textbf{\Large 卒業論文要旨 2019年度(令和元年度)}

\vspace{6.18mm}

\textbf{\Large ドライブレコーダーを利用した渋滞推定システム}
\end{center}

\vspace{10mm}

\begin{flushleft}
\textbf{論文要旨}\\
\end{flushleft}
日本における渋滞情報は主に道路に設置されているセンサーや監視カメラの映像等からリアルタイムに算出されており、そのような情報は自動車のナビゲーションシステム等で使われている。
しかしそのようなシステムは一部の道路に搭載されておらず、情報に限りがある。
この研究では、近年のドライブレコーダー搭載数の増加や映像技術向上を生かし、ドライブレコーダーから得られる単眼カメラ映像から渋滞推定を行うことで、全ての道路での渋滞推定を可能にすることを目指す。
実装にはドライブレコーダーから得られる単眼カメラ映像を用いて、距離推定ライブラリ及び物体検出システムを利用し、先行車との距離をもとに渋滞しているか否かを推定する。


\begin{flushleft}
\textbf{キーワード}\\
\textbf{ドライブレコーダー,渋滞,深層学習,画像処理}

\end{flushleft}

\begin{flushright}
\textbf{慶應義塾大学 環境情報学部}\\
\textbf{李 広耀}
\end{flushright}
\newpage

