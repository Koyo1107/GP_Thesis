\begin{center}
\textbf{\Large 卒業論文要旨 2020年度(令和2年度)}

\vspace{6.18mm}

\textbf{\Large Depth2Jam:ドライブレコーダーを用いた渋滞推定システム}
\end{center}

\vspace{10mm}

\begin{flushleft}
\textbf{論文要旨}\\
\end{flushleft}
日本において渋滞情報は主に道路に設置されているセンサーや監視カメラの映像から,逐次VICSなどの交通情報管理センターへ送られ,そこから自動車のナビゲーションシステム等に送信している.
しかし,現状渋滞情報を取得できるのは高速道路や国道などの一部の道路に限られており,そうでない道路では渋滞情報を収集できない.
また,ドライブレコーダーは近年の性能向上や危険運転や煽り運転の表面化の結果搭載率が高まっている.
しかし,搭載されたドライブレコーダーの映像を見るなどの活用をされるケースは少ない.
ドライブレコーダーは映像デバイスであるため物体検知や画像認識を行うことでさまざまなデーターを収集することが可能である.
本研究の目的は限られている渋滞情報が収集できるエリアを減らし,ドライブレコーダーの新しい活用方法を提案することである.
目的に対するアプローチとして,ドライブレコーダーの映像から渋滞推定を行うシステムDepth2Jamを提案する.
本研究ではDepth2Jamを実装し,渋滞推定精度について実験評価を行った.
また,Depth2Jamの精度について精度向上のための改善点と今後の展望について示した.

\begin{flushleft}
\textbf{キーワード}\\
\textbf{ドライブレコーダー,渋滞,深層学習,画像処理}

\end{flushleft}

\begin{flushright}
\textbf{慶應義塾大学 環境情報学部}\\
\textbf{李 広耀}
\end{flushright}
\newpage

