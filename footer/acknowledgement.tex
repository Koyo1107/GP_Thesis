\chapter*{謝辞}
本研究を進めるにあたり,ご指導を頂きました慶應義塾大学環境情報学部教授中澤仁博士に深く感謝いたします.
また,慶應義塾大学中澤研究室の諸先輩方には折りに振れ貴重なご助言を頂きました.
特に慶應義塾大学大学院政策・メディア研究科陳寅特任助教,慶應義塾大学大学院政策・メディア研究科大越匡特任講師には本論文を執筆するにあたってご指導頂きました.
ここに深く感謝の意を表します
そして、慶應義塾大学大学院博士課程 佐々木航氏、慶應義塾大学大学院博士課程 磯川直大氏,慶應義塾大学大学院修士課程 本木悠介氏、慶應義塾大学大学院修士課程 安井慎一郎氏には本研究に対し多くの時間を割いていただきご指導をいただきました。
特に慶應義塾大学大学院博士課程 佐々木航氏は、レインボーシックスシージを通じてさまざまな戦術を披露していただき、とても参考になりました。
また,安井慎一郎氏には卒論執筆に際してアドバイスをいただいたり、激励いただいたり、バーミヤンを奢っていただきました。ここに感謝します。

陰から研究活動を支えて頂いた、松尾さん、遠藤さんに深く感謝申し上げます.
そして、私のメンターである慶應義塾大学大学院修士課程 柿野優衣氏に深く感謝致します。
柿野氏にはWIPの時からお世話になりました。
特にTERMの時期には意見が噛み合わなかったり、発表前日に発表スライドを大幅に変えてしまったりと大変なご迷惑をおかけしました。
そして同じ研究室の先輩方の、慶應義塾大学大学院修士課程 谷村朋樹氏、慶應義塾大学大学院修士課程 川島寛乃氏、慶應義塾大学大学院修士課程 山田佑亮氏に深く感謝致します。

また、研究室の同期として、研究活動に切磋琢磨した野田悠加氏、スウィート哲也キース氏、橘直雪氏、マンタタ・タガツォ・ウイリアム氏、助川友里氏、菅原メリッサ紗良氏、鈴木航平氏、山根卓氏に感謝致します。
そして、違う分野ながらもお互いに卒論の執筆について切磋琢磨し、同時に私の精神面を支えてくださった彼女氏に深く感謝致します。
また、私がブレイクダンスをやらなくなってしまってからも積極的にダーツ等の遊びに誘っていただいた友人の岩間大輝氏に深く感謝致します。

最後に両親へ心から深く感謝を述べます。
前大学での1年半の学費及び生活費を支援してくれたにもかかわらず、中退してしまったことをこの場を借りて改めて謝罪します。
一度中退してしまった私に、半年の海外語学留学と、1年間の北九州予備校での浪人というチャンスをいただき、慶應義塾大学の塾生となることができました。
ここに深く感謝します。

\begin{flushright}
2020年1月21日\\
李 広耀
\end{flushright}
