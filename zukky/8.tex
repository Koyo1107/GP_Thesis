\chapter{結論}
本章では,本研究における今後の展望と本論文のまとめを述べる.

\section{本論文のまとめ}
本研究では,睡眠不足をゲーミフィケーションに適用する上で必要な要件を満たすために社会的ジェットラグの概念と計算手法を基にした睡眠評価手法を提案した.
睡眠は時間が多ければ多いほど良いという数値的特徴を持っておらず,一部のゲーミフィケーションに適用することが困難だった.
社会的ジェットラグの概念を基にした多ければ多いほど良いという数値的特徴をもつ睡眠評価手法を提案することでこの問題を解決した.
さらに提案した睡眠評価手法を用いてゲーミフィケーションを適用したiOSアプリケーション,Lag-Fitを実装して実験を行った.
実験は12/14〜1/10までの4週間で社会的制約が大きいグループ(千葉県立佐倉高等学校ラグビー部)と小さいグループ(徳田・高汐・中澤合同研究室Life-Cloudグループ)にそれぞれ5人ずつ,計10人で行った.
評価実験にて提案した睡眠評価手法は,社会的制約が大きいグループにおいて社会的ジェットラグと中程度の相関(r=-.440***)を持ちながら,一日単位で睡眠評価が行うことが可能でることを示した.
実験後,各ゲーミフィケーションごとに睡眠に対するモチベーションにどれだけつながったかをアンケートで定性的評価を行った.
結果は,「スコアとランキング」が五段階評価で最も高く3.7(最大5.0)となった.
また,実験期間中の社会的ジェットラグの変化を分析すると実験前と比べ第1週が平均約11分削減され,以降アプリケーションへのアクセスが減るにつれて増えていった.
本実験では10人を被験者としたが一般公開したアプリケーションから取得したデータを評価対象とすることでより精度の高い評価をとる.
実験期間も1か月より長い期間で睡眠が改善されるかを調査する必要があるだろう.
また,実験開始時にアプリケーションが開けないなどのトラブルがないことが望ましかった.
これらを本研究の課題として挙げる.

\section{今後の展望}
本節では,本研究で提案した睡眠評価手法およびLag-Fitアプリケーションの今後の展望についてまとめる.
\subsection{サービスとしての運用}
実験を通して判明した,アプリケーションの改善点を修正し,アプリケーションを実際にリリースする.
Androidへの対応やFitbit以外での睡眠検知をできるようにすることで多くの人が使用できるアプリケーションにする.
具体的には,ユーザ自身がグループを作成し,睡眠に対してゲーミフィケーションを適用できるようにする.

\subsection{評価精度向上}
本研究では就寝時間などのデバイスが検知できる睡眠データを用いたが,将来的に機械学習などで「快眠できたかどうか」などの主観的なデータの検知精度が向上し実用的になったとき,より精度の高い別の評価手法が提案できる可能性がある.
また,社会的制約や日中の生活パターンによる個人差も考慮して評価することで大学生などの社会的制約の小さい人の評価も行うことができる.

\subsection{睡眠以外へのライフログデータへの応用}
本研究では睡眠が多ければ多いほど良いという数値的特徴を持っていないことがゲーミフィケーション適用する上で難しい要素であることを述べた.
同じような特徴を持つライフログデータは多く存在する.
例えば,摂取カロリーなどが該当する.
その人が体重を増やしたいのか減らすしたいのかによって変わったり,最適な量も一人一人変わってくる.
そういったデータをゲーミフィケーションしやすい形に評価し直すことでより効果的な行動変容促進が可能になってくる.
また様々なライフログデータを評価し統合することでさらに広義の意味で健康をゲーミフィケーションすることが期待できる.
