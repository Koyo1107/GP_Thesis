\chapter{序論}
本章では,はじめに本研究における背景について述べる.
ついで,本研究の問題意識および目的を述べる.
最後に本論文の構成を示す.

\section{研究背景}
本節では,本研究の背景として現代の睡眠とライフログについて述べる.

\subsection{現代の睡眠}
現代人の睡眠時間の減少傾向が問題に挙げられる. 
特に高校生において,睡眠不足は深刻である.
平成24年度児童生徒の健康状態サーベイランス事業報告書\cite{Survey24}によると平成4年度の同調査と比べて男子高校生で28分,女子高生で26分の睡眠時間が減少している. 
寝起きの状況(表\ref{sleepmotivation})については「すっきり目が覚めた」と答えた者の比率が男子高校生18.7\%, 女子高校生16.3\%であった.
また,「最近、睡眠不足を感じている」と答えた者の比率は男子高校生57.5\%, 女子高校生66.0\%と男女共に半数を越える結果となった.
しかし,「最近、睡眠不足を感じている」と答えた人の「睡眠不足を感じる理由(複数回答,表\ref{reason})」では「なんとなく夜更かしをしてしまう」を選んだ人が男子高校生で46.1\%,女子高校生で48.0\%であった.
この調査結果から睡眠時間を確保する環境があるにも関わらず夜更かしをしてしまっていることが考えられる. 

\begin{table}[htbp]
	\begin{center}
		\caption{高校生の寝起きの状況}
		\label{sleepmotivation}
		\begin{tabular}{|c|ll|}
		\hline
		 & 男子高校生 & 女子高校生\\
		\hline \hline
		睡眠時間 & 6h36m & 6h23m \\
		睡眠不足を感じている人の比率 & 57.5\% & 66.0\% \\
		朝スッキリ起きられる人の比率 & 18.7\% & 16.3\% \\
		\hline
		\end{tabular}
	\end{center}
\end{table}

\begin{table}[htbp]
	\begin{center}
		\caption{睡眠不足を感じる理由(複数回答)}
		\label{reason}
		\begin{tabular}{|c|ll|}
		\hline
		 & 男子高校生 & 女子高校生\\
		\hline \hline
		宿題や勉強 & 47.7\% & 52.5\% \\
		\bfseries{なんとなく夜更かししてしまう} & \bfseries{46.1\%} & \bfseries{48.0\%} \\
		携帯,ネット,メール & 35.1\% & 41.7\% \\
		帰宅が遅い & 21.3\% & 19.6\% \\
		寝付けない & 18.0\% & 16.3\% \\
		テレビやDVD & 16.2\% & 11.4\% \\
		ゲーム & 15.3\% & 2.8\% \\
		家族が寝るのが遅いため & 5.7\% & 5.4\% \\
		その他 & 10.5\% & 9.8\% \\
		\hline
		\end{tabular}
	\end{center}
\end{table}

\newpage
\subsection{ライフログ}
ライフログとは,人間の生活や経験などを,デジタルデータとして記録する技術のことを指す.
ライフログには,ユーザ自身が操作して記録する手動記録と,外部デバイスから自動的に記録する自動記録がある.
本説では自動記録について詳細に説明する.
近年,スマートフォンやウェアラブルデバイス\cite{Fuelband}\cite{AndroidWear}の普及率が上昇している.
これらのデバイスの多くには加速度やGPS,ジャイロセンサなどの複数のセンサが搭載されている.
これによりユーザの様々なライフログデータを取得することが容易になった.
例えば,Moves\cite{Moves}は加速度センサ,およびGPSを用いてユーザがアクセスした場所と移動手段を検知しアプリケーションに記録する.
古川らはスマートフォンの加速度センサのみを用いてユーザの移動手段を判定するシステムBORO\cite{Boro}を提案し,消費電力の削減を行った.
他にも睡眠の時間や深さなどを様々な手法で測定することが可能である.
SleepBetterは加速度センサを用いてベッドのきしみを検知し,ユーザの睡眠の深さを検知する.
また時計型ウェアラブルデバイスの一種であるFitbit歩数や移動距離,心拍数,睡眠時間などのライフログデータを取得することが可能である.
専用スマートフォンアプリケーションとBluetooth経由で同期をすると,ユーザのライフログデータを可視化することが可能である.

%Movesの絵とか入れる?
%センサ一覧?

\begin{figure}[tbp]
	\begin{center}
		\begin{tabular}{cc}
			\begin{minipage}{0.5\hsize}
				\begin{center}
				\includegraphics[width=50mm]{figs/image/fitbitdevice.eps}
					\caption{Fitbitデバイス}
					\label{fitbit5}
  				\end{center}
  			\end{minipage}

			\begin{minipage}{0.5\hsize}
				\begin{center}
				\fbox{\includegraphics[width=30mm]{figs/image/moves.eps}}
					\caption{Moves}
					\label{fitbitsync}
				\end{center}
			\end{minipage}
		\end{tabular}
	\end{center}
\end{figure}


\section{本研究の問題意識}
ゲームの遊び自体のノウハウをゲーム以外の分野に活用することをゲーミフィケーションと呼ぶ.
ライフログデータをゲーミフィケーションに適用することでユーザの問題を解決するための行動変容を促すことが期待できる.
例えばFitbitアプリケーションでは歩数を「スコアとランキング」と呼ばれるゲーミフィケーションの要素を適用することでユーザがより歩くよう促進することが期待できる. 
歩数の値は5000歩の人より10000歩の人の方が良いなど,基本的に多ければ多いほど健康に貢献したと言える数値である.
従って,より上位を目指すことが目的となる「スコアとランキング」が行動変容に効果をもたらすことが期待できる.
このようにゲーミフィケーションをする上で数値的特徴とゲーミフィケーションの目的が一致した時に行動変容促進が期待できる.
しかし睡眠の場合,多ければ多いほど良いとは限らない数値の特徴を持っている.
例えば,6時間睡眠と9時間睡眠を比べた時に必ずしも後者が優れているとは言えない.
取得するべき睡眠時間は人によって異なり,平日の睡眠不足を補った際の長時間睡眠などを考慮しなければならない点があるためである.
また,「競争」などのゲーミフィケーションを用いる際に睡眠時間帯を知られることはプライバシーが大きく,共有することに抵抗がある人も多い.
本研究ではゲーミフィケーションに適した睡眠の評価手法の提案をし,これらの問題意識を解決を試みる.

\section{本研究の目的}
本研究の問題意識で述べたゲーミフィケーションを適用するための機能要件を満たした睡眠評価手法の提案および評価を行うことを目的とする.
本研究にて提案した睡眠評価手法とゲーミフィケーションを用いたアプリケーション「Lag-Fit」を被験者10人に使用してもらうことで評価する.

\section{本論文の構成}
本論文は,本章を含め全8章からなる.
本章では,本研究における背景と問題意識,目的を述べた.
第2章では,行動変容促進手法について整理し,睡眠に対する行動変容促進手法の問題意識について述べる.
第3章では,睡眠評価手法について整理し,本研究で提案する睡眠評価手法について述べる.
第4章では,本研究の関連研究について述べる.
第5章では,本研究で提案するアプリケーション「Lag-Fit」について述べる.
第6章では,本研究の評価実験の手法について述べる.
第7章では,評価実験の結果と考察について述べる.
第8章では,本論文の結論と今後の展望について述べる.
