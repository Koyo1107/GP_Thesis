\chapter{実験}
本章では実験の目的および方法について述べる.
\section{実験目的}
%評価手法の精度
%行動変容促進手法の効果
%社会的制約の強度と行動変容促進手法の効果の関係
睡眠評価手法の精度を評価する必要がある.
第3章3節でも述べた通り,睡眠評価手法に関して実験で明らかにしなければならないことは以下のようになる.
\begin{itemize}
	\item 社会的制約の強弱が評価精度にどう影響するのか.
	\item 社会的ジェットラグとの関係性が保てているか.
\end{itemize}
ついで睡眠評価手法を適用した各ゲーミフィケーションがどのように影響を与えたかを評価する.
最後にゲーミフィションによって睡眠がどのように変化したかを評価する.

\section{被験者}

被験者はiPhone所有者を対象としており,社会的制約が大きいグループ5人と社会的制約が小さいグループ5人,計10人を対象とした.
具体的には社会的制約が大きい人として千葉県立佐倉高等学校のラグビー部(以後,佐倉ラグビー部)のプレイヤー5人を対象とした.
このチームでは大体平日は朝から学校があるため,5時から7時頃の起床が必要である.
授業終了後,部活動があるため大体19時から20時程度まで拘束される.
従って,このチームに所属する人は社会的制約が大きいグループであると考える.
次に社会的制約が小さいグループとして慶應義塾大学 徳田・高汐・中澤合同研究室Life-Cloud(以下,Life-Cloud)の学部生5人を対象とした.
大学生は基本的に受けたい授業を履修するため,自身の活動時間帯に負荷がかかる時間の授業を避けやすい.
また,教師や先輩など社会的立場が上の人による強制力も少ないため,授業を欠席することへの抵抗も少ない.
従って,このチームに所属する人は社会的制約が小さいグループであると考える.
被験者の所属チームおよび平日の睡眠中央時刻(MSW),休日の睡眠中央時刻(MSF),睡眠調整MSF(MSFsc),社会的ジェットラグ(SJL)のデータを表\ref{user}に示す.

\begin{table}[htbp]
	\begin{center}
	\scalebox{1.0}{
	\begin{tabular}{|c|c|c|c|c|c|}
  	\hline
	User & 所属チーム & MSW & MSF & MSFsc & SJL \\
	\hline
	\hline
	User1 & \multirow{5}{*}{Life-Cloud} & 3:30 & 5:00 & 4:39 & 90 \\
	User2 &  & 4:30 & 5:00 & 3:56 & 30 \\
	User3 &  & 2:45 & 4:15 & 3:11 & 90 \\
	User4 &  & 5:00 & 6:00 & 6:00 & 60 \\
	User5 &  & 3:00 & 4:30 & 3:26 & 90 \\
	\hline
	User6 & \multirow{5}{*}{佐倉ラグビー部} & 3:00 & 4:00 & 3:39 & 60 \\
	User7 &  & 3:00 & 4:15 & 3:43 & 75 \\
	User8 &  & 3:00 & 3:00 & 3:00 & 0 \\
	User9 &  & 4:15 & 4:00 & 3:49 & 15 \\
	User10 &  & 2:45 & 4:00 & 3:49 & 75 \\
	\hline
 	\end{tabular}
	}
 	\end{center}
 	\caption{被験者のデータ}
 	\label{user}
\end{table}

\section{実験手順}
全ての被験者にFitbitデバイスを貸し出した.
FitbitアプリケーションとLag-Fitを各iPhoneにインストールしてもらった.
実験期間は4週間(12/14〜1/10)とした.
「週間ランキング」と「チーム戦」では月曜日を週の初めとして評価を行った.
実験用のLag-Fitではアプリケーションヘのアクセス時および目標時間変更時のタイムスタンプがサーバに記録される.
また,Fitbitの睡眠データ同期後にLag-Fitを開いた時「本日の睡眠を0〜100で自己採点してください!」とアラートが出現し,自己採点を要求する.(図\ref{alertquestionaire})
アンケート回答後,本システムが算出した得点を確認できる.

\begin{figure}[tbp]
	\begin{center}
		\includegraphics[width=70mm]{image/6/questionaire.eps}
		\caption{実験用Lag-Fitでのアラート画面}
		\label{alertquestionaire}
	\end{center}
\end{figure}

ラグビー部は部の予定表を参考に平日と休日の判定を行い,Life-Cloudは予めアンケートにて休日の日を申告してもらい判定をした.
また,第3週(12/28\textasciitilde1/3)は年末年始であることを踏まえて評価考察を行う.
被験者には毎日以下の3つの義務を与えた.
\begin{itemize}
	\item 就寝時Fitbitデバイスを着用して睡眠を取る.
 	\item 起床後Fitbitアプリケーションを開き同期をする.
	\item 同期後Lag-Fitを開き今日の睡眠を0〜100点で評価する.
\end{itemize}

実験の開始前に社会的ジェットラグの程度,終了後にアプリケーションの使用感のアンケートを取った.
また,毎週末に目標時間と睡眠不足の程度についてのアンケートをWeb上で取った.
実験前アンケート,毎週末アンケート,実験後アンケートの内容をそれぞれ表\ref{questionaire1},表\ref{questionaire2},表\ref{questionaire3}にまとめる.

\begin{table}[htbp]
	\begin{center}
	\scalebox{1.0}{
	\begin{tabular}{|c|c|}
  	\hline
	質問内容 & 質問形式 \\
	\hline
	\hline
	平日の大体の睡眠時間帯を教えてください. & 時間入力 \\
	休日の大体の睡眠時間帯を教えてください. & 時間入力 \\
	\hline
 	\end{tabular}
	}
 	\end{center}
 	\caption{実験前アンケートの内容}
 	\label{questionaire1}
\end{table}

\begin{table}[htbp]
	\begin{center}
	\scalebox{1.0}{
	\begin{tabular}{|c|c|}
  	\hline
	質問内容 & 質問形式 \\
	\hline
	\hline
	\multirow{2}{*}{\shortstack{現在のアプリ内で設定した目標時間はその日の睡眠不足を解消するのに十分な睡眠\\時間だと思いますか.}} & 5段階評価\\
	& \\
	途中で目標時間を変えた人は変更した理由を選択してください. & 選択肢 \\
	\hline
	今週の一日あたりのうたた寝や昼寝をした時間はどの程度ですか. & 時間入力 \\
	今週一週間の睡眠を0〜100点で評価してください. & 数値入力(0〜100)\\
 	\hline
	\end{tabular}
	}
 	\end{center}
 	\caption{毎週末アンケートの内容}
 	\label{questionaire2}
\end{table}

\begin{table}[htbp]
	\begin{center}
	\scalebox{1.0}{
	\begin{tabular}{|c|c|c|}
  	\hline
	章 & 質問内容 & 質問形式 \\
	\hline
	\hline
	\multirow{2}{*}{Fitbitについて} & 睡眠時のFitbitの着用に抵抗はありますか. & 時間入力 \\
	& Fitbitのつけ忘れの一番の原因は何ですか. & 時間入力 \\
	\hline
	\multirow{2}{*}{ユーザレベルについて} & ユーザレベルは睡眠改善の意識にどれだけ繋がりましたか. & 5段階評価\\
	& そう思う理由や感想を教えてください. & 自由記述\\
	\hline
	\multirow{2}{*}{実績画面について} & 実績は睡眠改善の意識にどれだけ繋がりましたか. & 5段階評価\\
	& そう思う理由や感想を教えてください. & 自由記述\\
	\hline
	\multirow{2}{*}{ランキング画面について} & ランキングは睡眠改善の意識にどれだけ繋がりましたか. & 5段階評価\\
	& そう思う理由や感想を教えてください. & 自由記述\\
	\hline
	\multirow{2}{*}{チーム戦画面について} & チーム戦は睡眠改善の意識にどれだけ繋がりましたか. & 5段階評価\\
	& そう思う理由や感想を教えてください. & 自由記述\\
	\hline
	\multirow{5}{*}{\shortstack{プライバシーについて\\ (データの種類)}} & アプリケーションが計算した得点の公開に抵抗はないですか& 5段階評価\\
	& 就寝時間の公開に抵抗はないですか & 5段階評価\\ 
	& 起床時間の公開に抵抗はないですか & 5段階評価\\
	& 睡眠時間の公開に抵抗はないですか & 5段階評価\\
	& 睡眠効率の公開に抵抗はないですか & 5段階評価\\
	\hline
	\multirow{7}{*}{\shortstack{プライバシーについて\\(公開相手)}} & 自分のチームメンバーへの得点の公開に抵抗はないですか &5段階評価\\
	& 他のチームへの得点の公開に抵抗はないですか &5段階評価\\
	& 先輩への得点の公開に抵抗はないですか &5段階評価\\
	& リーダー,上司への得点の公開に抵抗はないですか &5段階評価\\
	& 家族への得点の公開に抵抗はないですか &5段階評価\\
	& 医者への得点の公開に抵抗はないですか &5段階評価\\
	& TwitterやFacebookなどのSNSへの得点の公開に抵抗はないですか &5段階評価\\
	\hline
	\multirow{2}{*}{エピソードや会話} & 印象に残っているエピソード & 自由記述 \\
	& 記憶に残っているチーム内での会話 & 自由記述 \\ 
	\hline
	その他 & Lag-Fitにあれば良い機能を教えてください. & 自由記述 \\
	\hline
 	\end{tabular}
	}
 	\end{center}
 	\caption{実験後アンケートの内容}
 	\label{questionaire3}
\end{table}

\section{まとめ}
本実験における,目的を述べたあと,被験者と実験方法について述べた.
次章では実験の結果と考察をまとめる.
