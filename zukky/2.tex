\chapter{行動変容促進手法}
行動変容とは習慣化された行動パターンを変えることである.
本章では,人に行動変容を促進する際に考えられる行動変容の手法やモデルについて述べる.
その後,睡眠に関する行動変容促進手法の現状と問題意識についてまとめる.

\section{行動変容モデル}
一般的に人が健康などに良い行動をとるようになるには「危機感」「マイナス面よりプラスの面が大きい」の2つの条件が必要とされている.
例えば, 肥満による健康被害に対する「危機感」を持った人が運動することの手間(マイナス面)よりも健康になること(プラス面)の方が大切であると感じた時に運動に対する行動変容が起こる.
Prochaskaらが提唱した行動変容ステージモデル(Transtheoretical Model: TTM)\cite{TTM}は行動変容が起こる過程を「無関心期」「関心期」「準備期」「実行期」「維持期」の5ステージに分類したものであり, タバコの禁煙など,不健康な習慣を改善する時の行動変容過程の説明に用いられる.
具体的には「無関心期」は6ヶ月以内に行動を変えようと思っていない,「関心期」は6ヶ月以内に行動を変えようと思っている,「準備期」は1ヶ月以内に行動を変えようと思っている,「実行期」は行動を変えて6ヶ月未満である,「維持期」は行動を変えて6ヶ月以上であることを指す.
なお行動変容のプロセスは、常に「無関心期」から「維持期」に順調に進むとは限らない.「実行期」や「維持期」に入った後,行動変容する前のステージに戻る「逆戻り」という現象も起こる.

\section{ゲーミフィケーション}
ゲーム(主にテレビゲーム)の遊び自体のノウハウを,ゲーム以外の分野に活用することをゲーミフィケーションと呼ぶ.
「ゲーミフィケーション(Gamification)」は「ゲーム(game)」と「〜化する(fication)」の2つの言葉の造語であり,直訳すると「ゲーム化する」という意味合いを持つ.
ゲーミフィケーションは他の分野にゲームそのものを導入するわけではなく,人気あるゲームによく使われているユーザがゲームに熱中する要素を導入することを指す.
ゲーミフィケーション17の技術\cite{gamification17}(表\ref{gamification17table})は人が楽しくなる・夢中になる要素の代表的なものを17種類に体系化したものである.

\begin{table}[htbp]
	\begin{center}
		\caption{ゲーミフィケーション17の技術}
		\label{gamification17table}
		\begin{tabular}{|c|l|}
		\hline
		技術 & 内容\\
		\hline \hline
		即時フィードバック & 自分の行動に対して反応がすぐに返ってくる\\
		レベルアップ & 自分の強さや経験の量を明確に表す\\
		レベルデザイン & プレイヤーレベルに合わせた楽しみを用意\\
		不足感 & 「全部集めきった」という満足感を与える \\
		シークレット & 何が隠されているかわからないことで期待させる\\
		スコアとランキング &「自分の位置」を把握させて意欲を向上させる\\
		バッジと実績 & 利用者の達成度を可視化する\\
		競争 & 身近な相手の動きを知ってモチベーションをアップさせる\\
		協力 &「仲間と一緒にいたい」気持ちで継続させる\\
		価値観の共有 & ゲームに参加する人同士の交流を広げる\\
		ストーリー(物語性) & 覚えやすく記憶に残る物語を展開させる\\
		カスタマイズ & 自分のオリジナルキャラクターで愛着を持たせる\\
		イベント & 特別な出来事や催し物でワクワク感を高める\\
		リメンバー & 「期限付きの権利」で愛着心を高める\\
		プレリレーションシップ & 新作の発売に合せて前作も買ってもらう\\
		グラフィカル & 「絵」によって利用者は瞬時に楽しさを理解する\\
		驚嘆 & 利用者の想像を超えるサービス精神を発揮する\\
		\hline
		\end{tabular}
	\end{center}
\end{table}

これらの技術を使った活用事例としてSwarm\cite{Swarm}を挙げる.SwarmはFoursquare社が2014年にFoursquare\cite{Foursquare}アプリケーションからチェックイン機能を独立させたアプリケーションである.
Swarmではチェックイン回数を「競争」したり,アプリへの貢献度を「スコアとランキング」,各スポットごとに「バッジと実績」を設けるなど,複数のゲーミフィケーション手法を用いてユーザがアプリケーションを使い続ける工夫がなされている.
ゲーミフィケーションは行動変容促進にも多く活用されている.
ランニングの習慣管理アプリケーションNike+\cite{Nikeplus}は,SNSで「共有」することで友達からコメントをもらえたり,特定の条件をクリアするとメッセージが貰えるなどの仕組みを設けている.
FitbitアプリケーションではFitbitデバイス\cite{Fitbit}が検知した歩数データをスコアとランキング形式で表示する(図\ref{fitbitranking})ことでウォーキングを促進させる.

\begin{figure}[tbp]
	\begin{center}
		\includegraphics[width=50mm]{image/2/fitbitranking.eps}
		\caption{Fitbit歩数ランキング}
		\label{fitbitranking}
	\end{center}
\end{figure}

\section{睡眠に対する行動変容促進手法}
本節では,睡眠の行動変容促進手法への適用について整理する.
序論で述べたように現代社会では睡眠不足が蔓延している.
そのため,睡眠の管理や良い睡眠習慣をつけることを目的としたアプリケーションが多く存在する.
Fitbitアプリケーション(図\ref{fitbitsleep})では就寝時刻と起床時刻,睡眠時間などが可視化させ、ユーザが設定した目標睡眠時間を超えると右側に星マークがつく.
しかし,単純に睡眠時間の大きさのみを評価軸としているため12時間26分の睡眠時間を確保した時にも星マークがついてしまっている.
また,適した睡眠時間はユーザごとに大きく異なるため「競争」などの他者との比較を用いたゲーミフィケーションに応用することが難しい.
このように,睡眠不足に対しては一部のゲーミフィケーション手法を取り入れるのが難しいことが考えられる.

\begin{figure}[tbp]
	\begin{center}
		\includegraphics[width=50mm]{image/2/fitbitsleep.eps}
		\caption{Fitbitの睡眠ログ.目標時間を超えた時に星マークが付く.}
		\label{fitbitsleep}
	\end{center}
\end{figure}

\section{問題意識}
前節で述べたように,数値的理由により睡眠に対して一部のゲーミフィケーションを適用することが難しい.
従って,睡眠不足をゲーミフィケーションに適用させる上で睡眠不足の度合いを評価できる評価手法が必要である.
評価手法の必要用件などについては次章で詳細に述べる.

\section{まとめ}
本章では,行動変容促進手法とその一種であるゲーミフィケーションについてまとめた.
その後睡眠に対するゲーミフィケーションの現状と問題意識についてまとめた.
次章では,本研究でゲーミフィケーションを適用するための睡眠評価手法について述べる.
