\chapter{既存研究}
本章では,本研究に関連する既存研究を,睡眠検知,睡眠に関する行動変容促進手法の2節でまとめる.

\section{睡眠検知手法}
%
%
%
近年,スマートフォンやウェアラブルデバイスを用いた睡眠検知手法が存在する.
Toss 'N' Turn(TNT)\cite{TossN}では,睡眠状態や睡眠質(良い,悪い)をスマートフォンのみで分類するために,
音センサ,光センサ,加速度センサ,画面状態,アプリケーションの使用状況,バッテリーの状態(充電中であるか)を用いた.
TNTアプリケーションを被験者に使用してもらい,毎日睡眠の質を5段階で評価してもらう.
センサデータを独立変数,アンケートを従属変数として機械学習を行った.
具体的には,特徴選択を行い有意義な特徴をもつ独立変数のみを抽出した.
その後,決定木とベイジアンネットワークの2つの手法で評価した.
結果は,特徴選択を行った後ベイジアンネットワークを行った時が最も良く,93.06\%,個別のモデルでは94.52\%の精度で睡眠の良悪を判別できた.
アプリケーションのUIとシステム構成図を\ref{tossn}に示す.

\begin{figure}[tbp]
	\begin{center}
		\includegraphics[width=100mm]{image/5/tossN.eps}
		\caption{toss'N'turnのUI(左)とシステム構成図(右)}
		\label{tossn}
	\end{center}
\end{figure}

SleepBetter\cite{SleepBetter}(図\ref{sleepbetter})はAndroid,iOSアプリケーション用であり,就寝時にベットに置く事によって加速度の変化から睡眠の深さを検知する.
このアプリケーションは就寝前にユーザが睡眠を取る際に手動入力を求めることによって,睡眠の誤検知を減らす.
しかし,ユーザの手間や入力忘れが増える問題もある.

\begin{figure}[tbp]
	\begin{center}
		\includegraphics[width=60mm]{image/5/sleepbetter_log.eps}
		\caption{スマートフォンをベットの上に置く事で睡眠時間や質を検知するSleepBetter}
		\label{sleepbetter}
	\end{center}
\end{figure}

Abdullahら\cite{abdullah2014towards}は,スマートフォンの画面状態のみを用いてSocial JetLagを検知した.
Social JetLagを求めるには毎日の就寝時間と起床時間を検知できれば良い.
従って,睡眠の深さや質を用いる際に使用されやすい加速度センサや音センサなどを使用する必要がない.
画面状態のみを用いて睡眠時間帯を検知しているため低コスト,省電力で睡眠検知を行えることが大きな点である.


%\section{行動変容促進手法}
%tetujinさんの論文とか?
%placebo
%Syncrometer
%Choco
%いる?



\section{睡眠に関する行動変容促進手法}
%Shuteye
%Reverse&Alarm
%BuddyClock
睡眠に関しても睡眠不足を改善したり,より良い睡眠質を得るための日中活動に対する行動変容促進アプリケーションや既存研究が存在する.
Bauerらが提案したShutEye\cite{Shuteye}(図\ref{shuteye})は睡眠の質に影響するカフェインの摂取,うたた寝,運動,食事,飲酒,喫煙,リラクゼーションの7つの日中活動に行うべき時間帯があることに注目した.
ShutEyeはAndroid用アプリケーションで図のように背景画像に現在の時間(縦棒)に対して各習慣を行って良い場合太線,行っては行けない場合細線で可視化される.
良い日中活動の習慣がつくことが期待できるが,その習慣を行ったかどうかの判定が難しいため共有したり,他のゲーミフィケーションに適用することは難しい.

\begin{figure}[tbp]
	\begin{center}
		\includegraphics[width=120mm]{image/5/shuteye.eps}
		\caption{待ち受け画面で時間に応じて行っても良い習慣を可視化するShutEye}
		\label{shuteye}
	\end{center}
\end{figure}

Reverse\&Alarm Clock\cite{Reverse}(図\ref{reverse})は睡眠時間の確保を促すことができるAndroidの目覚ましアプリケーションである.
一般的な目覚ましは起床時間のみを指定するが,このアプリケーションでは目標の睡眠時間も入力する.
アプリケーションが起床時間と目標の睡眠時間から就寝するべき時刻を逆算して,時間が近くなるとアラートをする.
これにより,適切な睡眠時間の確保が期待できる.

\begin{figure}[tbp]
	\begin{center}
		\includegraphics[width=50mm]{image/5/reverse_bedtime.eps}
		\caption{Reverse\&Ararm Clockは就寝時間を通知する}
		\label{reverse}
	\end{center}
\end{figure}

ShiraziらはSNSに睡眠データを共有することが睡眠習慣に影響を与えるのかを検証した.
睡眠データをFacebook上に共有できるAndroidアプリケーションSomnometer\cite{shirazi2013already}を提案した.
Facebook上に共有するグループとそうでないグループにわけ比較実験を行った結果,共有するグループの方が高い睡眠評価を行った.

Kimらは友達やルームメイトなどのグループ内で睡眠状態を共有するBuddyClock\cite{kim2008you}を提案した.
ベッドの近くにBuddyClockを表示した端末を置いてしようする(図\ref{buddyclock1}).
BuddyClockはグループ内の現在の睡眠状態(Awake,Snoozing,Asleep)が可視化される\ref{buddyclock2}.
グループ内での親密度への変化と場所(同じ部屋,同じ建物,違う建物)の違いがどう影響を与えるかを実験した.
5つのグループで実験を行ったところ,共有情報はユーザ間の親密度に影響を与えた.
同グループの睡眠不足を心配する会話などが見られた.
また睡眠時間は平均9分増加した.

\begin{figure}[tbp]
	\begin{center}
		\begin{tabular}{cc}
			\begin{minipage}{0.5\hsize}
				\begin{center}
					\includegraphics[width=80mm]{image/5/buddyclockImage.eps}
					\caption{BuddyClockの使用イメージ.ベッド近くに置いて使用する.}
					\label{buddyclock1}
  				\end{center}
  			\end{minipage}

			\begin{minipage}{0.5\hsize}
				\begin{center}
					\includegraphics[width=50mm]{image/5/buddyclockUI.eps}
					\caption{BuddyClockのUI.グループ内の睡眠状態が可視化される.}
					\label{buddyclock2}
				\end{center}
			\end{minipage}
		\end{tabular}
	\end{center}
\end{figure}

\section{まとめ}
本章では,本研究に関連する既存研究を,睡眠検知,睡眠に関する行動変容促進手法の2節でまとめた.
