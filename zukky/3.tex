\chapter{睡眠評価手法}
本章では,はじめに問題の解決に必要な睡眠評価手法の必要用件について述べる.
既存の睡眠評価手法を整理する.
その後,本研究で用いる概日リズムやクロノタイプ,社会的ジェットラグの概念の説明および評価方法について述べる.
最後に今回当研究で提案する睡眠評価手法について説明する.

\section{睡眠評価手法の必要用件}
本節では睡眠不足に対してゲーミフィケーションを適用する上での睡眠評価手法について述べる.
前章で述べたように睡眠時間は多ければ多いほど良いという数値的特徴を持っていない.
この問題を解消する睡眠評価手法を提案する.
睡眠評価手法を提案する際に考えるべき点が複数挙げられる.
まず,第一に主観的な評価が評価の主軸に入らないことが挙げられる.
例えば,今日の快眠度合いなどを毎朝アンケートをとって評価することで睡眠不足度合いを考えることができる.
しかし,ユーザの主観を主軸とした睡眠評価手法を用いるとゲーム性が損なわれる可能性が高い.
本研究ではスマートフォンやウェアラブルデバイスなどで検知された睡眠を想定しているため,就寝時刻などのデバイスから取得できるデータを使って評価できる必要がある.
次に適切とされる睡眠時間が人によって異なることを考える.
例えば,適切な睡眠時間が異なる人同士で睡眠時間を競い合っても効果は期待できない.
よって適した睡眠時間がユーザごとに違うことを考慮できる必要がある.
最後に即時性について考える.
ゲーミフィケーション17の技術の1つに「即時フィードバック」があるように,ユーザが行ったことに対して即時に結果を見せることがゲームとしての面白さを提供する上で必要である.
つまり毎日の睡眠を一日単位で評価できる必要がある.
従ってゲーミフィケーションを適用する上での睡眠評価手法の必要用件は以下のようになる.
\begin{itemize}
\renewcommand{\labelenumi}{\arabic{enumi}.}
	\item 大きいほど良い,もしくは悪い数値的特徴を持つ.
	\item デバイスから取得したデータのみを使う.
	\item 適した睡眠時間の違いを考慮している.
	\item 睡眠不足を考慮している.
	\item 一日単位で評価ができる.
\end{itemize}
次節では,既存の睡眠評価手法についてまとめる.

\section{既存の睡眠評価手法}
睡眠の評価手法は今まで多く提案されてきた.
ピッツバーグ睡眠質問表(The Pittsburgh Sleep Quality Index: PSQI)\cite{PSQI}は睡眠障害の評価として広く使用されており, 睡眠の質, 入眠時間, 睡眠時間, 睡眠効率, 睡眠困難, 睡眠薬の使用,日中覚醒困難の7要素の合計得点で算出される. 
他にも2000年, 世界保健機構(WHO)との協同により世界睡眠・保健プロジェクトによって作成されたアテネ不眠尺度\cite{AIS}は,不眠症の度合いを過去 1 カ月間に少なくとも週3 回以上あてはまる症状項目の得点の合計点で評価する.
セントマリー病院睡眠質問表\cite{Stmary}は入院患者の睡眠評価のための質問票で,過去24時間の睡眠について14項目の質問によって評価する.
入院患者の日々の睡眠状況を評価するのに適している.
ロネンバーグらが2003年に提案したミュンヘンクロノタイプ質問紙(Munich ChronoType Questionnaire: MCTQ)\cite{MCTQ}は平日,休日それぞれの就寝時間,起床時間からクロノタイプ(睡眠時間帯)を評価する.
平日と休日の睡眠時間帯の差から睡眠不足の度合いがわかる.
ここで,MCTQは就寝時間と起床時間のみを評価の要素としていることに注目する.
就寝時間と起床時間は睡眠を検知するデバイスで検知できるものである.
他の睡眠評価手法は主観のアンケート結果の合計が評価が得点となるのに比べると定量的に評価が行える.
従って,MCTQはデバイスが検知した睡眠の評価に向いていると言える.

\section{ミュンヘンクロノタイプ質問紙}
%Circadian Rhythm
本説では,本研究で提案する睡眠評価手法の基となるミュンヘンクロノタイプ質問紙(Munich ChronoType Questionnaire: MCTQ)について説明する.
まず,概日リズム(Circadian Rhythm)について述べる.
概日リズムとは約24時間周期で変動する生理現象で,人類を含んだ動物や植物,菌類などほとんどの生物に存在している.
血圧、脈拍、深部体温、睡眠・覚醒サイクル、摂食行動、飲水行動、消化吸収、代謝など生物にとって基本的な生理学的、生化学的過程全てにみられる.
睡眠も約24時間周期のサイクルを持つ.
クロノタイプ(Chronotype)とは一般的に昼型夜型などと言われる物の事である.
人は昼行性で,通常は昼に活動的で夜に急速(睡眠)を取る動物であるが,そのタイミングは一人ひとり異なる.
この一人ひとりがもつ時間的なタイミングの傾向をクロノタイプと呼ぶ.
クロノタイプは,特別な用事がなければ,朝型の人は日の出ごろに目覚めて早い時間帯に活動的になる.一方夜型の人は日が高くなってから起きだし,深夜遅くに眠りにつく.
MCTQは,ロネンバーグらが2003年に提案した個人のクロノタイプを評価する質問紙である.
MCTQの大きな特徴として,睡眠不足によるクロノタイプのずれを考慮していることが挙げられる.
基本的にクロノタイプは仕事や学校と言った社会的制約のない休日の睡眠時間帯に反映されやすい.
しかし,睡眠不足が多くある現代では平日の睡眠時間が社会的制約によって減少してしまいがちである.
多くの人はその睡眠不足による負荷を,休日に多くの睡眠時間を確保することによって取り除く.
そのため, MCTQでは休日だけではなく平日との差分も考慮に入れてクロノタイプを算出している.
MCTQで用いるクロノタイプの算出方法を以下に示す.
MCTQでは「仕事がある日の就寝時間(SOw)」「仕事がある日の起床時間(SEw)」「仕事がない日の就寝時間(SOf)」「仕事がない日の起床時間(SEf)」「仕事がある日の1週間当たりの日数(WD)」を回答することで計算できる.
これらの数値から「仕事がある日の睡眠時間(SDw)」「仕事がない日の睡眠時間(SDf)」を以下の式から求められる.

\begin{eqnarray}
	SDw & = & SEw - SOw \\
	SDf & = & SEf - SOf
\end{eqnarray}

クロノタイプは人の睡眠時間帯のことである.
MCTQでは就寝時刻と起床時刻の中央値を睡眠時間帯として定義する.
例えば,就寝時刻が0時,起床時刻が6時の睡眠の場合,睡眠中央時刻が3時の睡眠であると言える.
従って「仕事がある日の睡眠中央時刻(MSW)」「仕事がない日の睡眠中央時刻(MSF)」は以下の式で求められる.

\begin{eqnarray}
	MSW & = & (SOw + SEw) / 2 \\
	MSF & = & (SOf + SEf) / 2
\end{eqnarray}

このMSFは比較的に人の睡眠時間帯に近い値となる.
MCTQは睡眠不足によるクロノタイプのずれを考慮する.
具体的には「1週間の平均睡眠時刻(SDweek)」を用いてMSFを修正した「睡眠調整MSF(MSFsc)」を以下の式にて求める.

\begin{eqnarray}
	SDweek & = & \{(SDw × MD + SDf × (7 - MD)\} / 7 \\
	MSFsc & = & MSF - (SDf - SDweek) /2
\end{eqnarray}

MSFscで算出された数値がその人のクロノタイプとなる.
例えば,MSFscが3時の場合,3時のクロノタイプを持つと言える.
表\ref{mctq}にて計算に必要な変数の略称および説明と計算式を示す.
表\ref{mctq_example}にて月曜日から金曜日に仕事があり睡眠負荷を抱える人の例を示す.

次に,同じくローネンバーグが2006年に提案した社会的ジェットラグ(Social Jetlag\cite{SocialJetlag})について述べる.
社会的ジェットラグとは社会的な時間と生物時計の不一致によって生ずる不調を指す.
多くの人は平日,仕事などの社会的制約によって睡眠不足を得て,その睡眠不足を補うために休日に長時間の睡眠時間を取ろうとする.
これにより生じた睡眠時間帯のズレにより起こる適用障害が時差ボケ(ジェットラグ)に似ていることから社会的ジェットラグと呼ばれている.
社会的ジェットラグは,平日の睡眠中央時刻と休日の睡眠中央時刻の差から算出する.
計算式は以下のようになる.

\begin{eqnarray}
	SJL & = & abs(MSF - MSW)
\end{eqnarray}

このSJLが少なければ少ない人程,クロノタイプが社会的活動に適用できていると言える.
月曜日から金曜日まで仕事がある人の例を表\ref{mctq_example}に示す.
例では平日の平均睡眠時間が5:06であるのに対して休日の平均睡眠時間が8:24である.
これは平日の睡眠不足を補うために休日で寝だめを行っていることが考えられる.
改善するためにはクロノタイプを早めるなどの工夫が必要である.
これまでに用いた略称と計算式を表\ref{mctq}にまとめる.
社会的ジェットラグはゲーミフィケーションに適用する上での必要要件を考えた時,一日単位で評価のみが不可能となっている.
次章では社会的ジェットラグの概念を用いた一日単位で評価が可能な睡眠評価手法を提案する.

\begin{table}[htbp]
	\begin{center}
	\begin{tabular}{|c|l|l|l|l|}
  	\hline
  	 & 就寝時刻 & 起床時刻 & 睡眠時間 & 中央時刻\\
	\hline
	 月 & 1:46 & 6:30 & 4h44m & 4:08 \\
	 火 & 1:28 & 6:34 & 5h6m & 4:01 \\
	 水 & 1:34 & 6:32 & 4h58m & 4:03 \\
	 木 & 1:00 & 6:38 & 5h38m & 3:49 \\
	 金 & 1:36 & 6:42 & 5h6m & 4:09 \\
	 土 & 2:28 & 10:46 & 8h18m & 6:37 \\
	 日 & 2:00 & 10:30 & 8h30m & 6:15 \\
	 \hline
	 平日 & 1:28(SOw) & 6:35(SEw) & 5:06(SDw) & 4:02(MSW) \\
	 休日 & 2:14(SOf) & 10:38(SEf) & 8:24(SDf) & 6:26(MSF) \\
	 \hline
	 概日リズム & 2:14 & 8:16 & 6:02(SDweek) & 5:15 (MSFsc) \\
	 \hline
	 社会的ジェットラグ & \multicolumn{4}{c|}{2:24(SJL)} \\
	 \hline
 	\end{tabular}
 	\end{center}
 	\caption{月曜日から金曜日まで仕事があり睡眠不足を抱える人の例}
 	\label{mctq_example}
\end{table}

\begin{table}[htbp]
	\begin{center}
	\begin{tabular}{|c|l|l|}
  	\hline
  	略称 & 説明 & 計算式\\
  	\hline
  	SOw & 仕事がある日の睡眠開始時刻 & \\
	SEw & 仕事がある日の起床時刻 & \\
	SDw & 仕事がある日の睡眠時間 & $ SEw - SOw $\\
	MSW & 仕事がある日の睡眠中央時刻 & $ (SOw + SEw) / 2 $\\
	SOf & 仕事がない日の睡眠開始時刻 & \\ 
	SOf & 仕事がない日の起床時刻 & \\
	SDf & 仕事がない日の睡眠時間 & $ SEf - SOf $\\
	MSF & 仕事がない日の睡眠中央時刻 & $(SOf + Sef) /2 $\\
	WD &  1週間当たりの仕事がある日の日数 & \\
	SDweek & 1週間あたりの平均睡眠時間 & $\{SDw × WD + SDf × (7 - WD)\} / 7 $\\
	MSFsc & 睡眠調整MSF & $ MSF - (SDf - SDweek)/ 2 $\\
	SJL & 社会的ジェットラグ & $abs(MSF - MSW)$ \\
	\hline
 	\end{tabular}
 	\end{center}
 	\caption{ミュンヘンクロノタイプ質問紙の計算式で使われる略語と説明および計算式\cite{mctqjp}}
 	\label{mctq}
\end{table}

\newpage

\section{提案する睡眠評価手法}
本節では,本研究で提案する睡眠評価手法について説明する.
MCTQを基に睡眠評価手法を作成する上で,その日の睡眠が社会的ジェットラグが削減される睡眠であるかを評価の主軸にした.
社会的ジェットラグは平日と休日の睡眠時間帯が同じ時に差が0となり一番良いとされる状態になる.
従って平日の睡眠が休日より少ない場合,平日の睡眠時間を増やす必要がある.

%平日の睡眠時間が足りないことが多いので平日の目標時間帯をアンケートで決めるよ
まず,目標とする平日の睡眠時間帯を計算する.
社会的ジェットラグが存在する大きな理由が平日の睡眠不足であるため,平日の取るべき睡眠時間を以下の2つの質問から判定する.

\begin{enumerate}
\renewcommand{\labelenumi}{\arabic{enumi}.}
	\item 平日の大体の就寝時刻(SOw)と起床時刻(SEw)を答えよ.
	\item 問1に加え,平日の睡眠時間はどの程度足りていない(LoS)と思うか?
\end{enumerate}

ここでは平日の定義を「目覚ましなどを使って起きる必要がある日」とした.
この質問は使用者のそれまでの状態を分析するための質問であり,評価したい睡眠との比較のみに用いられる.
「その日の快眠度」などの睡眠評価の主軸になる要素ではなく,毎日答える必要もない.
基本的に会社や学校などの社会的制約によって起床時刻が決められているため睡眠不足を感じる場合,就寝時刻を早める必要がある.
従って,目標とする就寝時刻(GSOw)と目標とする起床時刻(GSEw),目標睡眠時間(GSDw),目標睡眠時間帯(GMSW)を以下の式で求めた.

\begin{eqnarray}
	GSOw & = & SOw - LoS \\
	GSEw & = & SEw \\
	GSDw & = & GSEw - GSOw \\
	GMSW & = & (GSEw + GSDw) / 2
\end{eqnarray}
これにより目標とする睡眠時間帯がわかる.
目標睡眠時間帯が正しい時,毎日目標の睡眠時間帯で睡眠を取る事で社会的ジェットラグがなくなる.
%睡眠時間と睡眠効率と睡眠時間帯が大切であるよ
ゲーミフィケーションを適用する際に1日ごとに睡眠を評価できることが用件としてあるため,その日の睡眠が目標睡眠時間帯をどれだけ維持できる睡眠であるかを算出する.
睡眠評価をする上で睡眠時間,睡眠効率,睡眠時間帯の3つの要素が重要だと考えた.
%睡眠時間はこうやって計算するよ
まず,睡眠時間について考える.
当睡眠評価手法は睡眠不足を考慮する必要があるため,重要な要素であると言える.
社会的ジェットラグが存在する場合,平日の睡眠時間は目標睡眠時間よりも小さくなり,休日の睡眠時間は目標睡眠時間よりも大きくなる.
従って目標睡眠時間に近づけば近い程,社会的ジェットラグが小さい可能性が高い.
睡眠時間が0の時を0点,目標時間と等しい時に100点として評価する.
睡眠時間が目標睡眠時間より大きい場合は多い程評価は小さくなる.
評価される睡眠の就寝時間,起床時間,睡眠時間,中央時刻をそれぞれSOt,SEt,SDt,MSTとする.
睡眠時間の評価点(SDe)の計算式を以下のように算出する.
\begin{eqnarray}
	SDe & = &  SDt / GSDw (SDt < GSDw)\\
	SDe & = & 200 - SDt / GSDw (SDt > GSDw)
\end{eqnarray}
ただし,SDeが10点未満の時,10点とする.

%睡眠時間帯はこうやって計算するよ
次に睡眠時間帯について考える.
社会的ジェットラグは社会的制約を受けた生活リズムとクロノタイプが異なることが原因で起こる.
例えば休日十分な睡眠時間を取得できても平日のクロノタイプと異なる時間帯の睡眠である場合,平日の睡眠不足の改善が難しい.
睡眠時間帯も重要な要素であると言える.
平日の睡眠不足がどれだけ解決が見込める睡眠時間帯であるかを評価する.
睡眠時間帯の評価点(MSE)

\begin{eqnarray}
	MSE & = & 100 - (100 × abs(GMSW - MST) / GSDw)
\end{eqnarray}
ただし,MSEが10点未満の時,10点とする.

%睡眠効率はこうやって計算するよ
次に睡眠効率について考える.
睡眠効率とは睡眠時間として横になっている時間に対しての実際の睡眠時間の割合のことである.
睡眠時間帯を変えるフェイズでは今まで活動時間であった時間に寝ようとするため,睡眠が浅くなることが考えられる.
そのため,目標睡眠時間帯に睡眠を取ったにも関わらず寝不足を感じる場合がある.
睡眠効率はスマートフォンや多くのデバイスで検知することができ,Fitbitデバイスもその一つである.
従って当評価手法では睡眠効率(SQE)を用いる.
ただし,SQtが10点未満の時,10点とする.

%最終得点はこうやって計算するよ
最後に睡眠時間,睡眠時間帯,睡眠効率の3つを用いて評価する睡眠の得点を算出する.
睡眠の得点(SCORE)は以下の式で算出する.

\begin{eqnarray}
	SCORE & = & \sqrt[3]{SDe × MSE × SQt}
\end{eqnarray}

この数値は最大で100である.
値が大きい程良い数値であるため,競争などのゲーミフィケーションに適用することが可能である.
ゲーミフィケーションを適用する際の必要要件の適正を表\ref{reasonable2}に示す.

\begin{table}[htbp]
	\begin{center}
	\begin{tabular}{|c|c|c|}
  	\hline
  	 & 社会的ジェットラグ & 提案する睡眠評価手法\\
	\hline
	 大きい,小さい程良い数値的特徴を持つ & ○ & ○ \\
	 デバイスから取得したデータのみ使う& ○ & ○ \\
	 適した睡眠時間がユーザごとに違うことを考慮している & ○ & ○ \\
	 睡眠不足を考慮している & ○ & ○ \\
	 一日単位で評価ができる & × & ○ \\
	 \hline
 	\end{tabular}
 	\end{center}
 	\caption{社会的ジェットラグと提案する睡眠評価手法のゲーミフィケーションに適用する必要用件の適正}
 	\label{reasonable2}
\end{table}


%この評価手法の考察edit
%欠点として,目標時間が上手く決められるのかがわからない.
本研究で提案した睡眠評価手法は現実的に就寝時間を早める事が難しい人には適用できない.
例えば,帰宅が深夜で出社が早朝などの人はこれ以上就寝時間を早めるのが難しい.
また社会的ジェットラグの評価手法は社会的制約による寝不足を用いた計算手法のため,社会的制約の大きさの違いによる評価精度の違いも明らかにしなければならない.
最後に社会的ジェットラグを基にした評価手法であるため社会的ジェットラグとの関係性が保てているかを評価する必要があると言える.
%まとめ
\begin{table}[htbp]
	\begin{center}
	\begin{tabular}{|c|l|l|}
  	\hline
  	略称 & 説明 & 計算式\\
  	\hline
	Los & 平日の睡眠不足時間 & \\
	GSOw & 目標就寝時刻 & $ SOw - Los $\\
	GSEw & 目標起床時刻 & $ SEw $\\
	GSDw& 目標睡眠時間 & $ GSEw - GSOw $\\
	GMSW& 目標睡眠中央時刻 & $ (GSEw + GSDw) / 2 $\\
	SDe& 睡眠時間評価点& $ SDt / GSDw (SDt <= GSDw) $\\
	& & $200 - SDt / GSDw (SDt > GSDw) $\\
	MSE & 睡眠中央時刻評価点 & $ 100 - \{100 × abs(GMSW - MST)/GSDw\} $\\
	SQt & 睡眠効率 & \\
	SCORE & 睡眠評価点 & \(\sqrt[3]{SDe × MSE × SQt} \) \\
	\hline
 	\end{tabular}
 	\end{center}
 	\caption{本研究で提案する評価手法で使用される略語と説明および計算式}
 	\label{mctq}
\end{table}

\newpage
\section{まとめ}
本章では,既存の睡眠評価手法について整理した後,MCTQについて述べた.
ついで,本研究で提案する睡眠評価手法について述べた.

\if0
\begin{table}[htbp]
	\begin{center}
	\begin{tabular}{|c|l|l|l|l|}
  	\hline
  	 & 就寝時刻 & 起床時刻 & 睡眠時間 & 中央時刻\\
	\hline
	 平日 & 1:00 & 6:30 & 5h30m & 3:45 \\
	 \hline
	 平日の睡眠不足 & \multicolumn{4}{c|}{60m} \\
	 \hline
	 目標 & 0:00 & 6:30 & 6h30m & 3:15 \\
	 \hline
	 今日 & 1:00 & 6:30 & 5h30m & 3:45 \\
	 \hline
 	\end{tabular}
 	\end{center}
 	\caption{月曜日から金曜日まで仕事があり睡眠不足を抱える人の例}
 	\label{mctq_example}
\end{table}

\begin{table}[htbp]
	\begin{center}
	\begin{tabular}{|c|l|l|l|l|}
  	\hline
  	 & 睡眠時間(点) & 睡眠中央時刻(点) & 睡眠効率(点) & 睡眠評価点(点)\\
	\hline
	 評価点 & 84 & 92  & 90 & 88 \\
	 \hline
 	\end{tabular}
 	\end{center}
 	\caption{月曜日から金曜日まで仕事があり睡眠不足を抱える人の例}
 	\label{mctq_example}
\end{table}

\fi
