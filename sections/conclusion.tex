\chapter{結論}
ここでは,本研究における今後の展望と本研究のまとめについて述べる.
\section{今後の展望}
実験を通じて,ドライブレコーダーを使用することで渋滞を推定することが可能であることがわかった.
しかし,渋滞推定システムが実用化されるにはいくつか解決しなければならない問題がある.
\subsection{処理時間の問題}
課題のひとつに処理にかかる時間という問題がある.
現状のシステムはあらかじめ用意されたドライブレコーダー映像を処理しているが,Depth2Jamが実用化されるためにはリアルタイムで映像処理ができる必要がある.
しかし,現状のシステムでは1分の動画を処理するのに10分程度かかる.

処理時間の減少を図る方法はいくつか考えられる.
まず,処理する量を減らす手法である.現状のシステムでは深度推定ライブラリと物体検出ライブラリの実行を全ての映像のフレームで行っている.
この処理において,先に物体検出ライブラリで自動車類を検出し,検出できたフレームだけ深度推定ライブラリを実行するという方法である.
深度推定ライブラリを実行する回数を減らすことで総合的な処理時間を減少することが可能なのではないかと考えられる.

\subsection{容量の問題}
加えて車に搭載するには必要な容量が膨大である問題がある.
現状のシステムはTensorFlowライブラリとPyTorchライブラリの両方を利用しているため,実際に自動車に搭載してリアルタイムで検出するにはそれぞれのライブラリが入るほどの記憶容量が必要である.
しかし,現状のNvidia Jetsonといった開発向けのGPU装置はTensorFlowライブラリとPyTorchライブラリの両方を保存できるだけの容量の余裕がない.
この問題を解決する方法の一つとして使用するライブラリをTnsorFlowかPyTorchのどちらか一方に絞るという方法がある.
TensorflowもしくはPyTorchのどちらかに絞ることができれば容量削減と同時に呼び出すライブラリを減らすことができ,処理時間を減らすことが期待できる.


\section{本研究のまとめ}
本研究では新たな渋滞推定手法としてドライブレコーダー及び深度推定ライブラリ,物体検出ライブラリを用いた渋滞推定システムDepth2Jamについての提案を行った.
深度推定ライブラリを用いることでカメラからの距離を推量することが可能であり,物体検出ライブラリを用いることで推量したい物体を検出することが可能である.
実験では実際にドライブレコーダー映像をもとにDepth2Jamを用いて実証実験を行い,渋滞を推定できているか評価を行った.
評価実験においては評価指標にF値を用いた結果2割程度だった.
渋滞を正しく分類する精度はまだまだではあるが,さまざまな改善をすることでより精度向上が期待される.
