\chapter{評価}
ここでは本研究が行った実験における評価について述べる.

\section{実験の評価}
実験における結果のうち,定量評価である(渋滞的中率)を参照する.

\section{考察}
% 考察ってなんだよ -------------------
実験に用いた映像1から4は主に渋滞しているところをピックアップしたものであるのに対し,実験3で新たに加えた映像は渋滞していない状況が多く含まれていた.
そのため渋滞していると判断されていたものは多くは渋滞ではないという判断である確率が高くなっている.
新しく加えた映像データーは一般の道路を走行していたデーターであるため,本システムが実際に使われるためにはより渋滞の推定精度が向上される必要があると考えられる.(仮)

\section{今後の展望}
実験を通じて,ドライブレコーダーを使用することで渋滞を推定することが可能であることがわかった.
しかし,渋滞推定システムが実用化されるにはいくつか解決しなければならない問題がある.
\subsection{処理時間の問題}
課題のひとつに処理にかかる時間という問題がある.
現状のシステムはあらかじめ用意されたドライブレコーダー映像を処理しているが,本システムが実用化されるためにはリアルタイムで映像処理ができる必要がある.
しかし,現状のシステムでは1分の動画を処理するのに10分程度かかる.

処理時間の減少を図る方法はいくつか考えられる.
まず,処理する量を減らす手法である.現状のシステムでは深度推定ライブラリと物体検出ライブラリの実行を全ての映像のフレームで行っている.
この処理において,先に物体検出ライブラリで自動車類を検出し,検出できたフレームだけ深度推定ライブラリを実行するという方法である.
深度推定ライブラリを実行する回数を減らすことで総合的な処理時間を減少することが可能なのではないかと考えられる.

\subsection{容量の問題}
加えて車に搭載するには必要な容量が膨大である問題がある.
現状のシステムはTensorFlowライブラリとPyTorchライブラリの両方を利用しているため,実際に自動車に搭載してリアルタイムで検出するにはそれぞれのライブラリが入るほどの記憶容量が必要である.
しかし,現状のNvidia Jetsonといった開発向けのGPU装置はTensorFlowライブラリとPyTorchライブラリの両方を保存できるだけの容量の余裕がない.
この問題を解決する方法の一つとして使用するライブラリをTnsorFlowかPyTorchのどちらか一方に絞るという方法がある.
TensorflowもしくはPyTorchのどちらかに絞ることができれば容量削減と同時に呼び出すライブラリを減らすことができ,処理時間を減らすことが期待できる.

\subsection{精度向上のために}
現状のシステムではstruct2depthの色データーを取得して渋滞か否かを判断しているが,より精度を向上するにはstruct2depthで色を塗られる前の内部情報を取得する方法がある.
内部情報を取得して評価に組み込むことによって色分けして評価するよりもBOXを描写して色情報の平均を取るといった処理がなくなり,より少ない処理で評価まで辿り着ける上,色を塗られることなく評価が実現できる.
色を塗られることなく評価をすることができればより処理時間が少なくなり,また同時に高い制度が期待できる.