\chapter{はじめに}
本章では,はじめに本研究における背景とその目的等について述べる
\section{背景}
% 背景と問題----------
日本はアメリカや中国などの道路と比較して国土が狭く、人口密度が高い。それに伴って車道が狭く、渋滞が起こりやすくなっている。
そのため、日々テレビやラジオのニュースで渋滞情報が報道されている。

%渋滞を減らすことを目指すのではなく、渋滞を推定することのメリットについて語るべきかもしれない。-----

日本では、そのような渋滞情報は、VICS等の企業が道路に設置されているセンサーや車道付近に置かれているカメラの映像、および自動車に搭載されているセンサーやGPSの情報などをもとに算出されている。
そのような情報をもとに自動車のナビゲーションシステム等では到着予想時刻が概算される。
また、そのようなデーターはGoogle社が提供しているGoogle Mapにも提供されており、Web上でリアルタイムのデーターをいつでも閲覧することが可能である。
しかし、この渋滞情報は国道や高速道路といった、日常的に交通量が多い道路には設置されているが、そうではない、細い道や一方通行といった道では渋滞情報がなく、仮にそのような場所に車が密集すると予期しない渋滞に巻き込まれてしまうことが起こり得る。

%煽り運転の話------------軽く
また近年、煽り運転等のマナーの悪い運転が報道されるようになり、それに伴って各自動車へのドライブレコーダー搭載数が年々上昇している。
ドライブレコーダーは走行中の記録を撮ることにより事故の時の証拠を残すことができると同時に、事故等の証拠にもなり、警視庁も交通安全の面で、ウェブサイトにおいて取り付けを推奨している。
そもそも煽り運転等の危険運転はドライブレコーダーが登場する前から発生していたが、近年社会問題として取り上げられているのはドライブレコーダーやスマートフォンなど、映像技術の向上により人々が気軽に高画質な映像を録画、記録することが可能になったからである。
それに加えて近年のSNSの普及により、一般の人でも気軽に意見、画像、動画を発信できるようになった。
そのような背景もあり、煽り運転のような危険運転が容易に可視化され、ドライブレコーダーや、スマートフォンで撮影した映像がニュースでも取り上げられるようになった。
このような社会的変化を考えれば、ドライブレコーダーを単純な映像記録装置として扱うのでは不十分である。
さまざまな機能をつけることで、より利用価値を高める方がより効果的な使い方だといえる。

\section{目的}
%この研究の目指すところについて述べる----------軽く
この研究の目指すところは、入力された画像および動画から、走行している場所が渋滞しているか否かを判断することにある。
しかし、このシステムが目指すものは、その判断、評価のプロセスの次のフェーズとして、渋滞している場合、その実際に渋滞している現場の画像及び映像をサーバーにアップロードすること、さらにその後のフェーズとして、ドライバー同士でそのような渋滞情報を共有することを実装することである。
交通の安全上、ドライバーは運転中にカーナビ等の電子機器を操作することは禁止されているため、システムが渋滞だと判断した場合、自動でそのデータをアップロードする必要があり、また、その渋滞データーはオープンな情報としてドライバーに共有されることが必要である.
以上の2つのフェーズを最終的に加えることで、このシステムの本来目指しているものは完成する。
このシステムが実現すると、現在のVICS等の渋滞、交通量を検知、測定するシステムでは賄えないような、細い道、住宅街の道、一方通行など、さまざまな道路で突発的に発生した渋滞情報を共有することができる。

\section{ドライブレコーダーについて}
現在、一般的なドライブレコーダーは単なる記録装置として活用されているが、ドライブレコーダーの映像からは先行車や道路の状態、歩行者など、さまざまなデータを取得することが可能であり、このようなデーターを集めることで様々な事象を可視化することができることが考えられる。
この研究ではそのようなドライブレコーダーを使用して渋滞を推定することで、VICSやGoogle Map、そしてGPSの情報だけではでは見ることができない道路の渋滞の現場を可視化することを目的とする。
VICSでは車両感知器(Vehicle Detectors)と光学式車両感知器の2つが主に使われ、それぞれ通過車両台数や渋滞情報を自動的に感知し交通管制センターへ送信する役割と、通過車両を感知するとともに車載装置との双方向の通信を行う役割を持っている。
さらに、VICSは主に国産の自動車メーカーに取り付けられているカーナビゲーションシステムで主に利用されているため、外国産の車など、自動車によってはカーナビゲーションシステムがVICSに対応していない例もある。
そして、VICSは日本でのみ使われているシステムであるのに対し、ドライブレコーダーはどのような自動車でも追加で取り付けることが可能であり、渋滞推定を行うことができれば、日本に限らず世界中の道路で渋滞推定を行うことができる汎用性がある。

% 関連研究----------
% 渋滞の定義

\section{渋滞の定義}
渋滞についての研究を行うためには、まず渋滞とはどのような状態か定義する必要がある。
普段我々がニュース等で情報を得ている渋滞情報として、日本道路交通情報センター(JATIC)は、高速道路では、時速40km以下で低速走行あるいは停止発進を繰り返す車列が、1km以上かつ15分以上継続した状態であるとし、一般道では、時速10km以下で低速走行している状態が渋滞であると定義している。

この研究では正確な速度をカメラ映像から割り出すことは不可能なため、それらの定義とは異なる独自の定義を使用する。
%この研究で使っているGoogle Tensorflowが開発したStruct2depthシステムは画像及び動画のデーターを処理し、黄色と藍色の色の濃淡で距離を割り出している。
%カメラと位置が近ければ近いほど黄色い色が濃くなり、カメラから遠ざけば遠ざかるほど藍色の色が濃くなる。この研究では以上の状況を踏まえ、出力された画像及び映像から黄色い色の濃さの割合から先行車との車間距離を推定し、そこから速度を概算し、渋滞かどうか評価する。
%渋滞の定義においては、車道の自動車の数や交通量よりも走行速度の方に焦点が当たっている。


% 研究手法 でもここは別のところに書いた方がいいかもしれない


% 手法の工夫----------
% この工夫のパートが長すぎるかな...

% 実験----------

% 構成----------

