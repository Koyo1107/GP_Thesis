\chapter{はじめに}
本章では,はじめに本研究における背景とその目的等について述べる
\newpage
\section{背景}
% 背景と問題----------
日本はアメリカや中国などの道路と比較して国土が狭く、人口密度が高い。
それに伴って車道が狭く、渋滞が起こりやすくなっており、日々テレビやラジオのニュースで渋滞情報が報道されている。
渋滞の問題として、交通が滞ることによる物資や人員の運送の遅れだけでなく、ドライバーへの肉体的、精神的悪影響が挙げられる。
そのような背景のもと、ドライバーには渋滞情報をいち早く取得し、可能な限り渋滞を避けた運転をすることが求められている。

% 渋滞を減らすことを目指すのではなく、渋滞を推定することのメリットについて語るべきかもしれない。-----
日本では、渋滞情報は、VICS等の企業が道路に設置されているセンサーや車道付近に置かれているカメラの映像、および自動車に搭載されているセンサーやGPSの情報などをもとに算出されている。
また、日本におけるカーナビゲーションシステムはVICS等の渋滞情報を元に到着予想時刻などを割り出している。
VICSでは車両感知器(Vehicle Detectors)と光学式車両感知器の2つが主に使われ、それぞれ通過車両台数や渋滞情報を自動的に感知し交通管制センターへ送信する役割と、通過車両を感知するとともに車載装置との双方向の通信を行う役割を持っている。
加えて、渋滞情報はGoogle社が提供しているGoogle Mapにも提供されており、Web上でリアルタイムのデーターを閲覧することが可能である。
しかし、VICS等が管理している渋滞情報は、国道や高速道路といった日常的に交通量が多い道路には設置されているが、そうではない細い道や一方通行といった道では渋滞情報がない。

% 日本以外の交通情報について
日本以外の渋滞情報について、平成13年の警視庁によるトラフィック・インフォメーション・コンソーシアムでは、イギリス、ドイツ、アメリカの交通情報ビジネスについて述べられている\cite{traffic_buisiness}.
まずイギリスの交通情報ビジネスに関しては、Trafficmasterという企業が道路光津法の規定に基づき国からの免許を得て、さまざまな形で事業を展開している。
Trafficmasterの事業はイギリスだけにとどまらずドイツ、フランス、イタリア、オランダ、ベルギー等に及んでいる。
Trafficmasterは道路上に情報収集装置を設置し、赤外線センサーやカメラによって車両情報を読み取っている。
道路に情報収集装置を設置する点では日本における交通情報の取得と共通点がある。
次にドイツの交通情報ビジネスに関して、ドイツは高速道路網が発達しており、ダイムラー・クライスラーの子会社とドイツ・テレコムの共同出資によって設立されたTegaronや、イギリスのボーダフォングループに属するVodafone Tele Commerceによって交通情報ビジネスが行われている。
また、この両社が共同出資したDDGという企業は全国の高速道路に自ら設置した情報収集装置のほか、協力企業等の車両、警察や道路管理者等からの情報を統合し、上記の2社に情報提供を行なっている。
そしてアメリカでは、複数の民間事業者が道路交通情報ビジネスを行なっており、特にSmart Route Systemsという企業は全米21の大都市圏で事業を展開している。
日本を含めて、これら3カ国の道路交通情報の取得に関して共通しているのはどれも道路に情報収集装置を設置しているという点である。

\newpage

\section{ドライブレコーダーについて}
近年、煽り運転等のマナーの悪い運転が報道されるようになり、伴って各自動車へのドライブレコーダー搭載数が年々上昇している。
ドライブレコーダーは走行中の記録を撮ることにより事故の時の証拠を残すことができると同時に、事故等の証拠にもなり、警視庁も交通安全の面で、ウェブサイトにおいて取り付けを推奨している。
元々、煽り運転等の危険運転はドライブレコーダーが登場する前から発生していたが、近年社会問題として取り上げられているのはドライブレコーダーやスマートフォンなど、映像技術の向上により人々が気軽に高画質な映像を録画、記録することが可能になったからだと考えられる。
加えて近年のSNSの普及により、一般の人でも容易に事故の動画を発信したりと、注意喚起を行うことが可能になった。
結果、煽り運転のような危険運転が容易に可視化され、ドライブレコーダーや、スマートフォンで撮影した映像がニュースでも取り上げられるようになった。

% ドライブレコーダーの使用頻度の話を軽く ------------------
一般的なドライブレコーダーは事故の記録等で使われている。
しかし現状、ドライブレコーダーの搭載が増加しているにもかかわらず、そのほとんどの用途が事故等の、もしものための貯蔵となっている。
国土交通省の統計によると、ドライブレコーダーを取り付けた人のうち、実際にレコーダーの映像を見直すといった活用をしているのは2割ほどという結果が出ている\cite{ministryofland}。
ドライブレコーダーの映像からは先行車や道路の状態、歩行者など、さまざまなデータを取得することが可能であることを考えると、現状のドライブレコーダーを単純な映像記録装置として扱うのでは不十分であり、さまざまな機能をつけることで、より効果的、効率的に扱えるようになると言える。

\section{目的}
%この研究の目指すところについて述べる----------軽く
本研究の目指すところは、ドライブレコーダーから入力された画像および動画から、走行している場所が渋滞しているか否かを判断することにある。
ドライブレコーダーを使用して渋滞を推定することで、VICSやGoogle Map、そしてGPSの情報だけではでは見ることができない道路の渋滞の現場を可視化することが可能である。
VICSなどの道路に設置された情報収集装置と異なり、ドライブレコーダーはどのような自動車でも追加で取り付けることが可能であり、ドライブレコーダーで渋滞推定を行うことができれば、日本に限らず世界中の道路で渋滞推定を行うことができる汎用性がある。

加えて、本研究が目指すものは、渋滞の判断、評価のプロセスの次のフェーズとして、渋滞している場合、実際に渋滞している現場の画像及び映像をサーバーにアップロードすること、さらにその後のフェーズとして、ドライバー同士でそのような渋滞情報を共有することである。
交通の安全上、ドライバーは運転中にカーナビ等の電子機器を操作することは禁止されているため、システムが渋滞だと判断した場合、自動で渋滞データをアップロードする必要があり、また、渋滞データーはオープンな情報としてドライバーに共有される必要がある。
%以上の2つのフェーズを最終的に加えることで、このシステムの本来目指しているものは完成する。

\newpage

\section{渋滞の定義}
渋滞についての研究を行うためには、まず渋滞とはどのような状態か定義する必要がある。
普段我々がニュース等で情報を得ている渋滞情報として、日本道路交通情報センター(JATIC)は、高速道路では、時速40km以下で低速走行あるいは停止発進を繰り返す車列が、1km以上かつ15分以上継続した状態であるとし、一般道では、時速10km以下で低速走行している状態が渋滞であると定義している。
本研究では正確な速度をカメラ映像から割り出すことは不可能なため、それらの定義とは異なる独自の定義を使用する。
評価基準については5章の予備実験3の項で詳しく述べる。
%この研究で使っているGoogle Tensorflowが開発したStruct2depthシステムは画像及び動画のデーターを処理し、黄色と藍色の色の濃淡で距離を割り出している。
%カメラと位置が近ければ近いほど黄色い色が濃くなり、カメラから遠ざけば遠ざかるほど藍色の色が濃くなる。この研究では以上の状況を踏まえ、出力された画像及び映像から黄色い色の濃さの割合から先行車との車間距離を推定し、そこから速度を概算し、渋滞かどうか評価する。
%渋滞の定義においては、車道の自動車の数や交通量よりも走行速度の方に焦点が当たっている。

% 構成----------
\section{構成}
本論文の構成は以下の通りである。
2章で本稿で述べた問題について深く述べ、3章で問題にまつわる関連研究について述べる。
4章では本研究におけるアプローチとその設計と実装について述べる。
そして5章で本研究で実装したシステムに関して予備実験を行い、6章で本実験について述べる。
最後に7章で本実験に関して評価および考察を行い、8章で本研究のまとめについて述べる。