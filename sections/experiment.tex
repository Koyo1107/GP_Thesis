\chapter{実験}
ここでは本研究において行った実験について述べる.

\section{実験1 - さまざまな状況におけるDepth2Jamの推定精度実験}
\label{sec:exp1}
実装と予備実験の項で述べた実験を通してDepth2Jamをさらに改良し,さまざまな状況下のドライブレコーダー映像を使ってDepth2Jamの推定精度を測定する実験を行った.
これまでの実験に用いたデーターセットはどれも前方に先行車があるという状況だったが,この実験では先行車がない映像や信号で停止している映像を用いてDepth2Jamの推定精度に関して実験評価を行う.
実験3を行うにあたって人間の目で渋滞しているフレーム数を数える必要があるため,Depth2Jamにおいて出力された映像の中央にフレーム数を明記されるように改良した.
また,本実験においては人間の目で渋滞を判断するにあたって,信号等の要素を排除し,車間距離が近いと感じられる,あるいは車道において先行車が停止し,停止ランプがついており,ドライブレコーダーが取り付けられている自動車も停車している状況を渋滞と判断し,渋滞だと判断したフレーム数をGround Truthとした.

\section{実験2 - Depth2Jamの精度評価実験}
\label{sec:exp2}
Depth2Jamにおける再現性,適合率及びmAP値を測定するための実験を行った.
実験にて出力された映像データー5つから,それぞれ20フレームずつ抽出し,合計100フレーム画像において,Depth2Jamの渋滞判断においてさまざまな指標における結果を\figref{tab:mAP_fig}に示す.
表における総合とは,それぞれの映像から合計20フレーム画像ずつ抽出し,合計100画像となったテストデーターを指す.
また,TP,FP,FN,TNのそれぞれの説明および評価指標の説明を以下に示す.
\begin{itemize}
  \item TP : Depth2Jamが渋滞だと判断し,実際に渋滞しているフレーム
  \item FP : Depth2Jamが渋滞だと判断したが,実際には渋滞していないフレーム
  \item FN : Depth2Jamが渋滞ではないと判断したが,実際には渋滞していたフレーム
  \item TN : Depth2Jamが渋滞でないと判断し,実際に渋滞していないフレーム
\end{itemize}
加えて,比較対象として,画面全面の自動車類を検出した場合の結果を\tabref{tab:mAP_fig_noweght}に示す.

\subsection{使用したデータセット}
本実験において使用したデータセットを以下の表に示す.

% ----------------------------------------------------
\begin{table}[htbp]
  \centering
  \begin{scriptsize}
  \begin{tabular}{cccccc}
  \toprule
映像番号 & 映像の内容 & 映像時間 & フレーム数 & 時間帯 & 道路 \\
  \midrule
映像1 & 全く渋滞していない & 60秒 & 1830フレーム & 昼 & 一般道 \\
映像2 & 先行車あり スムーズに進んでいる & 60秒 & 1830フレーム & 昼 & 一般道 \\
映像3 & 途中信号による停車あり & 60秒 & 1830フレーム & 昼 & 一般道 \\
映像4 & 渋滞中の継続的な渋滞 & 60秒 & 1830フレーム & 昼 & 一般道 \\
映像5 & 信号による停車 & 31秒& 928フレーム & 昼 & 一般道 \\
  \bottomrule
  \end{tabular}
  \end{scriptsize}
  \caption{実験 データーセット}
  \label{tab:exp_dataset3}
\end{table}
% ------------------------------------------------------

\chapter{実験結果と考察}
データセットで述べたドライブレコーダー映像においてDepth2Jamが出力した結果および人間の判断の結果を\tabref{tab:exp3_fig}に示す.
\tabref{tab:exp3_fig}の通り,映像1の結果は先行車がなく,全く渋滞していないドライブレコーダー映像であり,Depth2Jamにおける出力結果においても渋滞の誤検出はなかった.
それに対して,映像2においては,先行車ありではあるが止まることなくスムーズに進んでいる自動車のドライブレコーダー映像である.
スムーズに進んでいても,4フレームにおいて渋滞だと誤検出するケースが発生した.
映像3~5に関してはどの映像においてもGround Truthに対してDepth2Jamにおける渋滞推定フレーム数は少なくなっている.
% ----------------------------------------------------
\begin{table}[htbp]
  \centering
  \begin{scriptsize}
  \begin{tabular}{cccc}
  \toprule
映像番号 & 映像の内容 & Depth2Jamの渋滞推定フレーム数(a) & Ground Truth(b)\\
  \midrule
映像1 & 先行車なし 無渋滞 & 0 & 0 \\
映像2 & 先行車あり スムーズに進んでいる & 4 & 0 \\
映像3 & 途中信号による停車あり & 167 & 870\\
映像4 & 渋滞中の継続的な渋滞 & 130 & 452\\
映像5 & 信号による停車 & 251 & 927 \\
\bottomrule
\end{tabular}
\end{scriptsize}
  \caption{実験1 - 結果}
  \label{tab:exp3_fig}
\end{table}
% ------------------------------------------------------


% ----------------------------------------------------
\begin{table}[htbp]
  \centering
  \begin{scriptsize}
  \begin{tabular}{cccccccccc}
  \toprule
映像 & TP & FP & FN & TN & 正解率(Accuracy) & 適合率(Precision) & 再現率(Recall) & 特異度(Specificity) & F値(F-measure) \\
  \midrule
映像1 & 0 & 0 & 0 & 100 & 1 & 0 & 0 & 1 & 0 \\
映像2 & 0 & 0 & 0 & 100 & 1 & 0 & 0 & 1 & 0 \\
映像3 & 7 & 0 & 43 & 50 & 0.57 & 1 & 0.14 & 1 & 0.25 \\
映像4 & 4 & 0 & 29 & 67 & 0.71 & 1 & 0.12 & 1 & 0.22 \\
映像5 & 26 & 0 & 74 & 0 & 0.26 & 1 & 0.26 & 1 & 0.33 \\
総合 & 5 & 0 & 31 & 64 & 0.69 & 1 & 0.14 & 1 & 0.24 \\
\bottomrule
\end{tabular}
\end{scriptsize}
  \caption{実験2 結果}
  \label{tab:mAP_fig}
\end{table}
% ------------------------------------------------------


% ----------------------------------------------------
\begin{table}[htbp]
  \centering
  \begin{scriptsize}
  \begin{tabular}{cccccccccc}
  \toprule
映像 & TP & FP & FN & TN & 正解率(Accuracy) & 適合率(Precision) & 再現率(Recall) & 特異度(Specificity) & F値(F-measure) \\
  \midrule
映像1 & 0 & 8 & 0 & 92 & 0.92 & 0 & 0 & 0.92 & 0 \\
映像2 & 0 & 29 & 0 & 71 & 0.71 & 0 & 0 & 0.71 & 0 \\
映像3 & 10 & 5 & 43 & 42 & 0.52 & 0.33 & 0.19 & 0.89 & 0.24 \\
映像4 & 7 & 15 & 29 & 49 & 0.56 & 0.32 & 0.19 & 0.77 & 0.24 \\
映像5 & 69 & 0 & 31 & 0 & 0.69 & 1 & 0.69 & 0 & 0.82 \\
総合 & 11 & 10 & 15 & 64 & 0.75 & 0.52 & 0.42 & 0.86 & 0.46 \\
\bottomrule
\end{tabular}
\end{scriptsize}
  \caption{(参考)画面全体を検出した場合の精度}
  \label{tab:mAP_fig_noweght}
\end{table}
% ------------------------------------------------------

\section{評価}
本研究では,実装したシステムの精度を評価するにあたってF値を用いた.
F値とは,再現率と適合率の調和平均の値であり,渋滞している/していないのような混合行列となるシステムの評価に適しているものである.
\tabref{tab:mAP_fig}の表の通り,映像1および映像2において渋滞していると判断できるフレームは存在しないのでF値は0となっているが,映像3~5及び総合フレームにおいては0.22~0.33という結果になった.
正解率をみると0.26~1とばらつきがあるものの,F値が0.2付近にあるため,現段階のDepth2Jamの渋滞推定精度はまだ実用化には遠く,性能向上が必要であると評価される.

